\documentclass{./../Latex/handout}
\usepackage[colorlinks=true]{hyperref}

\begin{document}
\thispagestyle{plain}
\myheader{How to install R?}

To install R:
\begin{itemize}
\item Open page: \href{https://www.r-project.org}{https://www.r-project.org}
\item 
Click on the link “download R” found in the third sentence of the first paragraph on the page.
\item It will ask you to choose a mirror. Choose a United States mirror.
\item The next page will provide links that match your OS (Windows, Mac, or Linux), click on the link that corresponds to your OS and install the most recent version depending on your OS. %\item For Windows Select “Install R for the first time”, this will download a .exe file. Once the .exe is downloaded, click and run the Installer. For Macintosh Select “R-4.2.1” either for “Big Sur” or “High Sierra”.
\end{itemize}

The RStudio IDE is a set of integrated tools designed to use R more easily. It includes a console,
syntax-highlighting editor that supports direct code execution, and a variety of robust tools for
plotting, viewing history, debugging and managing the workspace. We will be doing all of our work
in RStudio (there is hardly ever a reason not to).

After installing R, you also need to install RStudio. You can do so by using this \href{https://www.rstudio.com/products/rstudio/download/#download}{link}. You need the RStudio Desktop (free).


\end{document}