\documentclass{./../handout}

\fancyhf{}
\lhead{ECON 340: Economic Research Methods}
\rhead{Div Bhagia}
\rfoot{\center \thepage}

\begin{document}
\thispagestyle{plain}
\begin{center}
  \framebox{
    \vbox{
    \vspace{1em}
     \Large \sc Confidence Intervals and Hypothesis Testing \\
     \vspace{0.5em}
      \small ECON 340: Economic Research Methods \hfill Instructor: Div Bhagia
      }
    }
\end{center}	
\underline{Confidence Intervals} 

\textit{Known population variance:} \\
$1-\alpha$ confidence interval for the population mean $\mu$:
$$ \bar{X} \pm  \underbrace{z_{\alpha/2}  \frac{\sigma}{\sqrt{n}}}_{\text{Margin of Error}} $$
where $z_{\alpha/2}$ is the $z$-value that leaves area $\alpha/2$ in the upper tail of the standard normal distribution. 

\textit{Unknown population variance:} \\
$1-\alpha$ confidence interval for the population mean $\mu$:
$$ \bar{X} \pm  t_{n-1,\alpha/2}  \frac{S}{\sqrt{n}} $$
where $t_{n-1,\alpha/2}$ is the $t$-value that leaves area $\alpha/2$ in the upper tail of the t-distribution. $n-1$ is the degrees of freedom. Since $t$ distribution looks just like the standard normal for large $n$, for $n \geq 100$ continue using the standard normal table. \\~\\

\textit{Exercise.} A car manufacturer wants to estimate the mean CO2 emissions of a new model of car. A sample of 196 cars is randomly selected and their CO2 emissions are measured. The sample mean and standard deviation are 120 g/km and 20 g/km, respectively. Construct a 95\% confidence interval for the true mean CO2 emissions of this car model.



\newpage
\underline{Hypothesis Testing} 

Test null hypothesis $ H_0: \mu = \mu_0 $ against alternative hypothesis $H_1: \mu \neq \mu_0$. Construct test statistic $Z$ if true population variance is known, else use $T$-statistic.  
 $$ z_0 = \frac{\bar{x}-\mu_0}{\sigma/\sqrt{n}} \quad \text{and} \quad t_0 = \frac{\bar{x}-\mu_0}{S/\sqrt{n}}  $$
Under the null $\bar{X} \sim N(\mu_0, \sigma^2/n)$, then $Z \sim N(0,1)$ and $T \sim t_{n-1}$. 
In case of known population variance, reject the null if $|z_0|>z_{\alpha/2}$. In the case of unknown population variance, reject the null if $|t_0|>t_{n-1,\alpha/2}$. When $n \geq 100$ you can reject the null if $|t_0|>z_{\alpha/2}$. \\


\underline{p-value:} 

Known variance: $ p = 2Pr(Z>|z_0|)$ \\
Unknown variance, $n<100$: $p=2Pr(T>|t_0|)$ \\
Unknown variance, $n \geq 100$:  $p=2Pr(Z>|t_0|)$ \\

\textit{Exercise.} The car manufacturer had initially claimed that the average CO2 emissions of this model would be 115 g/km. Test the manufacturer's claim at a 5\% level of significance. What is the $p$-value associated with your test statistic?


\end{document}