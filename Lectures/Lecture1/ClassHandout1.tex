\documentclass{./../../Latex/handout}


\fancyhf{}
\lhead{ECON 340: Economic Research Methods}
\rhead{Div Bhagia}
\rfoot{\center \thepage}

\begin{document}
\thispagestyle{plain}
\begin{center}
  \framebox{
    \vbox{
    \vspace{1em}
     \Large \sc Summation Notation \\
     \vspace{0.5em}
      \small ECON 340: Economic Research Methods \hfill Instructor: Div Bhagia
      }
    }
\end{center}	

The capital sigma ($\Sigma$) stands for summing everything on the right. 
$$ \sum_{i=1}^N X_i = X_1 + X_2 + ... + X_N $$

\underline{Things you CAN do to summations:}
\begin{enumerate}
\item Pull constants out of them, or into them.
$$ \sum_{i=1}^N b X_i = b \sum_{i=1}^N X_i  $$
\item Split apart (or combine) sums (addition) or differences (subtraction)
$$ \sum_{i=1}^N (b X_i + c Y_i) = b \sum_{i=1}^N X_i  + c \sum_{i=1}^N Y_i $$
\item Multiply through constants by the number of terms in the summation
$$ \sum_{i=1}^N (a+b X_i)= aN + b \sum_{i=1}^N X_i  $$
\end{enumerate}

\underline{Things you CANNOT do to summations:}
\begin{enumerate}
\item Split apart (or combine) products (multiplication) or quotients (division).
$$ \sum_{i=1}^N X_i Y_i \neq  \sum_{i=1}^N X_i \times \sum_{i=1}^N Y_i   $$
\item Move the exponent out of or into the summation.
$$ \sum_{i=1}^N X_i^a \neq  \left(\sum_{i=1}^N X_i\right)^a $$
\end{enumerate}

\underline{Exercise}:
$$ X = \{2,9,6,8,11,14\} \quad \quad Y = \{7,1,3,5,0\}$$ 
\begin{enumerate}
\item $\mathlarger{\sum}_{i=1}^4 X_i = $ \\~\\
\item $\mathlarger{\sum}_{i=1}^4 2 X_i = $ \\~\\
\item $\mathlarger{\sum}_{i=1}^4 (X_i+4) = $ \\~\\
\item $\mathlarger{\sum}_{i=1}^3 (X_i+Y_i) = $ \\~\\
\item $\mathlarger{\sum}_{i=1}^2 X_i Y_i = $ \\~\\
\item $\mathlarger{\sum}_{i=1}^2 X_i \times \mathlarger{\sum}_{i=1}^2 Y_i = $ \\~\\
\item $\mathlarger{\sum}_{i=1}^2 X_i^2 $ \\~\\
\end{enumerate}

\end{document}