\documentclass{./../../Latex/handout}
\usepackage{tasks}
\begin{document}
\thispagestyle{plain}
\myheader{Handout for Lecture 20 \\ \normalsize{Calculus Review and Functional Forms}}
\rhead{Class Handout for Lecture 20}
\vspace{-1em}

For a function $ y = f(x) $, the derivative denoted by $dy/dx$ or $f'(x)$ captures how the value of the function changes due to a small change in $x$.

\underline{Rules of differentiation:}
\begin{tasks}(2)
\task $\displaystyle y=a  \rightarrow  \frac{dy}{dx}=0$
\task $\displaystyle y=ax \rightarrow \frac{dy}{dx}=a$
\task $\displaystyle y=ax^b \rightarrow  \frac{dy}{dx}=abx^{b-1}$
\task $\displaystyle y=f(x) \pm g(x) \rightarrow  \frac{dy}{dx}=f'(x) \pm g'(x)$ 
\task Derivative of a log function:
$$ y=log(x)  \rightarrow  \frac{dy}{dx}=\dfrac{1}{x}$$
\task Chain rule:
$$z=f(y),\ y=g(x) \rightarrow  \frac{d z}{d x}=\dfrac{d z}{d y} \cdot \frac{d y}{d x}$$
\end{tasks}
\vspace{-2em}

\textit{Find the derivative for the following functions:}
\begin{enumerate}
	\item $ \displaystyle y = 10 \rightarrow  \frac{dy}{dx}=0$
	\item $ \displaystyle y = 5x \rightarrow  \frac{dy}{dx}=5$
	\item $ \displaystyle y = 8x^3 \rightarrow  \frac{dy}{dx}=24x^2$
	\item $ \displaystyle y = 3x^2 + 4 \rightarrow  \frac{dy}{dx}= 6x$
	\item $ \displaystyle y = 2 + 3 \cdot log(x)\rightarrow  \frac{dy}{dx}= \frac{3}{x} $
	\item $ \displaystyle y = log(z),\ z = x^2 \rightarrow  \frac{dy}{dx}= \frac{dy}{dz} \cdot \frac{dz}{dx} = \frac{1}{z}\cdot 2x = \frac{2x}{x^2} = \frac{2}{x}$
	\item $ \displaystyle y = log(x^2) \rightarrow  \frac{dy}{dx}= \frac{1}{x^2}\cdot 2x = \frac{2}{x}$
	\item $ \displaystyle y = log(f(x)) \rightarrow  \frac{dy}{dx}= \frac{f'(x)}{f(x)}$
\end{enumerate}

\newpage 
\begin{itemize}
	\item 
Find $\frac{dY}{dX}$ for the following model:
$$ Y = \beta_0 + \beta_1 X + \beta_2 X^2 + u $$
What is the interpretation of $\beta_1$ and $\beta_2$?
$$ \frac{dY}{dX}  = \beta_1 + 2 \beta_2 X $$

\(\beta_1\) is the marginal effect of \(X\) on \(Y\) when \(X\) is equal to 0. 

\(\beta_2\) captures the change in the marginal effect of \(X\) on \(Y\) for each additional unit of \(X\). In other words, for each one-unit increase in \(X\), the effect of \(X\) on \(Y\) changes by \(2\beta_2\). \\

\item Consider the following model:
$$ \log(Y) = \beta_0 +  \beta_1 \log(X) + u$$
Differentiate both sides of the above equation with respect to $X$ and show that \(\beta_1\) represents the elasticity of \(Y\) with respect to \(X\). 

Differentiating the left-hand side, by chain rule we get:
\[ \frac{d}{dX} \log(Y) = \frac{1}{Y} \cdot \frac{dY}{dX} \]
Differentiating the right-hand side :
\[ \frac{d}{dX} (\beta_0 + \beta_1 \log(X) + u) = \beta_1 \cdot \frac{1}{X} \]

Setting the derivatives equal to each other since we are differentiating the same function:
\[ \frac{1}{Y} \cdot \frac{dY}{dX} = \beta_1 \cdot \frac{1}{X} \]
Re-arranging the terms, we get:
\[ \beta_1 = \frac{X}{Y} \cdot \frac{dY}{dX} \equiv \text{Elasticity of \(Y\) with respect to \(X\)}  \]
\end{itemize}

\end{document}