\documentclass{./../../Latex/handout}
\begin{document}
\thispagestyle{plain}
\myheader{Handout for Lecture 14 \\ \normalsize{Hypothesis Testing \& p-values}}
\rhead{Class Handout for Lecture 14}

\vspace{-0.5em}
\underline{Hypothesis Testing} \vspace{-0.5em}
\begin{enumerate}
  \item Set up null hypothesis and  alternative hypothesis  
 \begin{align*}
  	\text{Null Hypothesis: } \quad & H_0: \mu = \mu_0 \\
  	\text{Alternative Hypothesis: } \quad & H_1: \mu \neq \mu_0
  \end{align*}

  \item Construct test statistic $Z$ if true population variance is known, else use $T$-statistic.  
 $$ z_0 = \frac{\bar{x}-\mu_0}{\sigma/\sqrt{n}} \quad \text{and} \quad t_0 = \frac{\bar{x}-\mu_0}{S/\sqrt{n}}  $$
 \item Under the null if $\bar{X} \sim N(\mu_0, \sigma^2/n)$, then $Z \sim N(0,1)$ and $T \sim t_{n-1}$. 
In case of known population variance, reject the null if $|z_0|>z_{\alpha/2}$. In the case of unknown population variance, reject the null if $|t_0|>t_{n-1,\alpha/2}$.
\end{enumerate}
\vspace{-0.5em}
\textit{Note}: When $n \geq 100$ you can reject the null if $|t_0|>z_{\alpha/2}$ as in large sample $t$ distribution looks like the standard normal. 

\underline{p-Value:} 

p-Value is the probability of obtaining an outcome even more surprising under the null hypothesis than the one you got. \vspace{-0.5em}
\begin{itemize}
  \item Known variance: $ p = 2Pr(Z>|z_0|)$ 
\item Unknown variance, $n<100$: $p=2Pr(T>|t_0|)$ 
\item Unknown variance, $n \geq 100$:  $p=2Pr(T>|t_0|)=2Pr(Z>|t_0|)$ 
\end{itemize}

\newpage
\textit{Question 1:} A car manufacturer wants to estimate the mean CO2 emissions of a new model of car. A sample of 196 cars is randomly selected and their CO2 emissions are measured. The sample mean and standard deviation are 120 g/km and 20 g/km, respectively. The car manufacturer had initially claimed that the average CO2 emissions of this model would be 115 g/km. Test the manufacturer's claim at a 5\% level of significance.

\textit{Answer:} Null and alternative hypothesis:
 \begin{align*}
  	 H_0: \mu = 115 \quad \quad H_1: \mu \neq 115 
  \end{align*}
  Note that here, $\bar{x}=120$, $S=20$, and $n=196$. We can calculate the t-statistic as follows:
  $$ t_0 = \frac{\bar{x}-\mu_0}{S/\sqrt{n}} =  \frac{120-115}{20/\sqrt{196}} = 3.5 $$
Under the null $T \sim t_{195}$. However, since the sample size is large enough, we can just use the normal distribution to find the critical value. Critical value: $z_{0.025} = 1.96$ leaves 2.5\% area in the upper tail. Since $|t_0|=3.5>1.96$ we will reject the null at 5\% level of significance.  

\textit{Question 2:} Find the $p$-value associated with your test statistic in the previous question.

\textit{Answer:} Here the $p$-value is given by:
$$p=2Pr(T>|t_0|)=2Pr(Z>|t_0|) = 2Pr(Z>3.5) = 2 \times 0.002 = 0.004$$ 
So only 0.4\% outcomes for the sample mean would be more surprising than 120 g/km that we found if the true population mean was indeed 115 g/km.
\end{document}