\documentclass{./../../Latex/handout}
\begin{document}
\thispagestyle{plain}
\myheader{Handout for Lecture 15 \\ \normalsize{Ordinary Least Squares \& Goodness of Fit}}
\rhead{Class Handout for Lecture 15}

\underline{Simple Linear Regression Model:} $\quad Y = \beta_0 + \beta_1 X + u $
\begin{itemize}
    \setlength{\itemsep}{0pt}
  \item $Y$: Dependent variable (outcome or response variable)  \item $X$: Independent variable (explanatory variable, regressor)
  \item $\beta_0, \beta_1$: intercept and slope (population parameters)
  \item $u$: mean zero error term, $E(u)=0$
\end{itemize}

\underline{Ordinary Least Squares (OLS)} 

To obtain estimates for the intercept and slope of the line, we minimize the distance between the fitted line and the sample data. Let $X_i$ and $Y_i$ denote the $i$'th observation of $X$ and $Y$ in the sample data. \vspace{-0.5em}
\begin{itemize}
    \setlength{\itemsep}{0pt} 
  \item $\hat{Y}_i$: predicted value of $Y_i$
  \item $\hat{\beta}_0, \hat{\beta}_1$: OLS estimators for the intercept and slope
  \item Residuals/error: $ \hat{u}_i = \hat{Y}_i -Y_i$ (Note that we can always write $ Y_i = \hat{Y}_i + \hat{u}_i $)
\end{itemize}
OLS estimators for the intercept $\hat{\beta}_0$ and slope $\hat{\beta}_1$ are obtained by minimizing the sum of squared residuals: $$ \sum_{i=1}^n \hat{u}_i^2  = \sum_{i=1}^n (Y_i - \hat{Y}_i)^2   $$

\underline{A Measure of Goodness of Fit}
$$ R^2 = \frac{ESS}{TSS} = 1-\frac{RSS}{TSS} $$ 
where $$ TSS = \sum_{i=1}^n (Y_i-\bar{Y})^2, \quad \quad ESS = \sum_{i=1}^n (\hat{Y}_i-\bar{Y})^2, \quad  \quad RSS = \sum_{i=1}^n (Y_i-\hat{Y}_i)^2 =\sum_{i=1}^n \hat{u}_i^2$$

\newpage
\underline{Example: Predicting Final Exam Scores} \\~\\

You've collected data on monthly revenue ($Revenue_i$) and advertising spending ($AddSpend_i$) for several months for a small business. You fit the following line using OLS: 
\[ \hat{Revenue}_i = 50 + 3 \cdot AddSpend_i, \quad \quad R^2 =0.65 \]
\begin{enumerate}
    \item What are the estimated intercept and slope in the given fitted line?
    $$\hat{\beta}_0 = 50 \hspace{4cm} \hat{\beta}_1 = 3 \hspace{3cm} $$
    \item Interpret the intercept and slope coefficient. \\
    \textit{Intercept}: Predicted revenue is \$50 when advertising expenditure is 0.\\
    \textit{Slope}: Predicted revenue increases by \$3 for every \$1 increase in advertising expenditure.
    \item What is the predicted revenue for a month where the advertising spending was \$50?
       $$ 50 + 3 \cdot 50 = \$200 $$
    \item  If in a particular month, the revenue was \$100 and advertising spending was $20$, what would be the residual \( \hat{u} \) for this observation? 
    $$\hat{Revenue} = 50 + 3 \cdot 20 = 110, \quad \quad  \hat{u} = Revenue- \hat{Revenue} = 100-110=-10  $$
    \item  How does the predicted revenue increase due to an increase of \$10 in advertising spending? 
    $$ 3 \times 10 = \$ 30 $$
        
    \item  What percentage of the variability in revenue is explained by advertising spending? \\
    65\% since $R^2 = 0.65$
    \newpage
    \item (A bit challenging, try at home.) If I tell you, the variance of revenue over months is $125$. Can you tell me what is the variance of advertising spending? 
    
    Note we are given $$R^2 =Var(\hat{Revenue}) $$
     $$R^2 = \frac{Var(\hat{Revenue})}{Var(Revenue)} = 0.65 \rightarrow Var(\hat{Revenue}) = 0.65 \times 125 = 81.25  $$
     \[ \hat{Revenue}_i = 50 + 3 \cdot AddSpend_i \rightarrow  Var(\hat{Revenue}) = 3^2 \cdot Var(\hat{AddSpend}) \]
     The above follows from $Var(a + bX) = b^2 Var(X)$. So here $Var(\hat{AddSpend})=81.25/9 = 9.03$.
    \end{enumerate}


\end{document}