\documentclass{./../../Latex/handout}
\begin{document}
\thispagestyle{plain}
\myheader{Handout for Lecture 13 \\ \normalsize{Confidence Intervals}}
\rhead{Class Handout for Lecture 13}

How to construct a confidence interval? 

 \underline{Known population variance:} $1-\alpha$ confidence interval for the population mean $\mu$:
$$ \bar{x} \pm  z_{\alpha/2} \cdot \frac{\sigma}{\sqrt{n}} $$
where $z_{\alpha/2}$ is the $z$-value that leaves area $\alpha/2$ in the upper tail of the standard normal distribution.

\underline{Unknown population variance:} $1-\alpha$ confidence interval for the population mean $\mu$:
$$ \bar{x} \pm  t_{n-1,\alpha/2}  \frac{S}{\sqrt{n}} $$
where $t_{n-1,\alpha/2}$ is the $t$-value that leaves area $\alpha/2$ in the upper tail of the t-distribution. $n-1$ is the degrees of freedom. 

\textit{Note}: Since the $t$ distribution looks just like the standard normal for large $n$, for $n\geq 100$ you can continue using the standard normal table. \\

\newpage
\textit{Question:} A car manufacturer wants to estimate the mean CO2 emissions of a new model of car. A sample of 196 cars is randomly selected and their CO2 emissions are measured. The sample mean and standard deviation are 120 g/km and 20 g/km, respectively. Construct a 95\% confidence interval for the true mean CO2 emissions of this car model. (Note: $Pr(Z>1.96)=0.025$.)

\textit{Answer:} We are given:
\begin{align*}
\bar{x} &= 120 \, \text{g/km} \quad &\text{(sample mean)} \\
S &= 20 \, \text{g/km} \quad &\text{(sample standard deviation)} \\
n &= 196 \quad &\text{(sample size)} 
\end{align*}
We can use the following formula to create a confidence interval:
$$ \bar{x} \pm  t_{n-1,\alpha/2}  \frac{S}{\sqrt{n}} $$
Since we want to create a 95\% confidence interval, here $1-\alpha=0.95$. In which case, the $T$ statistic we want is $t_{195,0.025}$. However, since the degrees of freedom are large enough, $t_{195,0.025} \approx z_{0.025}=1.96$. So the 95\% confidence interval is given by:

\[
 120 \pm 1.96 \left( \frac{20}{\sqrt{196}} \right) =  [117.2, 122.8]
\]
Therefore, we are 95\% confident that the true mean CO2 emission of this car model is between 117.2 g/km and 122.8 g/km.


\end{document}