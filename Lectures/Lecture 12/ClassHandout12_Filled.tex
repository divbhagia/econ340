\documentclass{./../../Latex/handout}
\begin{document}
\thispagestyle{plain}
\myheader{Handout for Lecture 12 \\ \normalsize{Good Estimators, Sample Mean Distribution, and Confidence Intervals}}
\rhead{Class Handout for Lecture 12}

\fbox{\begin{minipage}{\textwidth}
\underline{Good Estimators} 

Denote $\hat{\theta}$ as an estimator for the population parameter $\theta$. Some desirable properties for an estimator \vspace{-1em}
\begin{itemize}
\normalsize
\item \textit{Unbiasedness}: $ E(\hat{\theta}) = \theta $
\item \textit{Efficiency}: lower variance is better
\item \textit{Consistency}: as the sample size becomes infinitely large,  $\hat{\theta} \rightarrow \theta$ 
\end{itemize}
\end{minipage}} \\~\\

\textit{Question 1:} If some sample estimator $\hat{\theta}$ is an unbiased estimator for the true population parameter $\theta$ i.e. $E(\hat{\theta}) = \theta$. This implies that: \vspace{-1em}
\begin{itemize}
\item[$\square$] $\hat{\theta}=\theta$ in all samples.
\item[$\boxtimes$] If we take repeated samples, average of $\hat{\theta}$ is equal to $\theta$ \\
\end{itemize}

\textit{Question 2}: We are choosing between two estimators $\hat{\theta}_1$ and $\hat{\theta}_2$, both of which are unbiased i.e. $E(\hat{\theta}_1)=\mu$ and $E(\hat{\theta}_2)=\mu$. But the variance of $\hat{\theta}_1$ is lower than that of $\hat{\theta}_2$ i.e. $Var(\hat{\theta}_1)<Var(\hat{\theta}_2)$.  Which of the following is true? \vspace{-1em}
\begin{itemize}
\item[$\square$] We are indifferent between the two estimators.
\item[$\boxtimes$] We prefer $\hat{\theta}_1$ over $\hat{\theta}_2$.
\item[$\square$] We prefer $\hat{\theta}_2$ over $\hat{\theta}_1$.
\item[$\square$] We need more information to reach any conclusion. \\~\\
\end{itemize}

\newpage
\fbox{\begin{minipage}{\textwidth}
\underline{Sample Mean Distribution} 

Let $X_1,X_2,...,X_n$ denote independent random draws (random sample) from a population with mean $\mu$ and variance $\sigma^2$. Then the sample mean 
$ \bar{X} = \frac{1}{n} \sum_{i=1}^n X_i $ is a random variable with:
$$E(\bar{X}) = \mu \quad \quad  Var(\bar{X}) = \frac{\sigma^2}{n} $$ 
In addition, the distribution of the sample mean is \textit{normal} if \textit{either} of the following is true: \vspace{-1em}
\begin{itemize}
  \item The underlying population is normal
  \item The sample size is large, say $n\geq 100$ 
\end{itemize}
Given the variance of the sample mean as \(\sigma_{\bar{X}}^2 = \frac{\sigma^2}{n} \), its standard deviation, commonly referred to as the \textit{standard error}, is \( \sigma_{\bar{X}} = \frac{\sigma}{\sqrt{n}} \). 
\end{minipage}} \\

\textit{Question 3}: If the average of hourly wages in the population is \( \mu = \$30 \) and the variance of hourly wages is \( \sigma^2 = 16 \). Then what is the expected value, variance, standard error, and distribution of the sample mean estimator for a sample size of 400? 

$$ E(\bar{X}) = 30, \quad \sigma^2_{\bar{X}} = \frac{16}{400} = 0.04, \quad \sigma_{\bar{X}} = \sqrt{0.04} =0.2 $$
Since $n \geq 100$, by Central Limit Theorem $ \bar{X}$ is normally distributed. \\

\textit{Question 4}: You are interested in the average starting salary of CSUF graduates and are considering taking a random sample of 120 students. I advise you to take as large of a sample as feasible. This is sound advice because taking an even larger sample would ensure that \vspace{-1em}
\begin{itemize}
\item[$\square$] The sample average $\bar{x}=\mu$
\item[$\square$] The sample average $\bar{x}$ is drawn from a normal distribution
\item[$\boxtimes$] The sample average $\bar{x}$ is drawn from a distribution with lower variance 
\end{itemize}
\textit{Note}: I am using $\bar{x}$ to denote a realization of $\bar{X}$. \\

\newpage
\textit{Question 5}: Can you explain intuitively why the variance of the sample mean increases with $\sigma^2$ and decreases with $n$? \\

\textit{Why does the variance of the sample mean decrease with $n$?} 

Suppose you aim to determine the average test score for all students at a university. When you use smaller samples, you could encounter significant sample-to-sample variability. For example, one sample might consist of students who performed exceptionally well, while another might include students who scored poorly. In contrast, a larger sample is more likely to accurately represent the overall student population, thereby reducing the variability between different samples. \\

\textit{Why does the variance of the sample mean increase with $\sigma^2$?}

If the range of scores is quite wide, one sample could consist of students who performed exceptionally well, leading to a high sample mean, while another might include students who scored poorly, resulting in a low sample mean. However, if there is little to no variation in the student test scores across the university, you are likely to obtain similar sample means across different samples.

\newpage
\fbox{\begin{minipage}{\textwidth}
\underline{Confidence Intervals}\\

Let $z_{\alpha/2}$ be the $z$-value that leaves area $\alpha/2$ in the upper tail of the normal distribution. \\
Then $1-\alpha$ confidence interval is given by 
$$ \bar{x} \pm  \underbrace{z_{\alpha/2}  \frac{\sigma}{\sqrt{n}}}_{\text{Margin of Error}} $$
\end{minipage}} \\

\textit{Question 6}: Continuing with Question 3, say you took a sample of 400 individuals and found the average hourly wages in your sample of $\bar{x} = 26$. Create a 95\% confidence interval for the true population mean. 

Note that here $1-\alpha = 0.95$, so $\alpha/2=0.025$. From the Standard Normal Table, $z_{0.025}=1.96$. In which case, the 95\% confidence interval is given by:
$$ 26 \pm 1.96 \times 0.2 = [25.6, 26.4] $$

 


\end{document}