\documentclass{./../handout}

\fancyhf{}
\fancyheadoffset{1cm}
\lhead{ECON 340: Economic Research Methods}
\rhead{Div Bhagia}
\rfoot{\center \thepage}

\begin{document}
\thispagestyle{plain}
\begin{center}
  \framebox{
    \vbox{
    \vspace{0.1em}
     \Large \sc Handout for Lecture 3 \\ \vspace{0.1em}
     \large Variance, Standard Deviation, Z-Score \\
     \vspace{0.5em}
      \small ECON 340: Economic Research Methods \hfill Instructor: Div Bhagia
      }
    }
\end{center}	

\begin{enumerate}
\item We surveyed a group of 100 individuals to determine their preference for ice cream. Among the respondents, 70 individuals expressed a liking for ice cream, while the remaining 30 individuals reported not liking it. To represent this data, we introduced a variable denoted as $X$, assigned a value of 1 to individuals who enjoy ice cream, and a value of 0 to those who do not.  \\~\\
Use the frequency distribution table to calculate the mean and variance of $X$. Here are the formulas you will need.
$$ \bar{X} = \sum_{k=1}^K f_k X_k \quad \quad \quad \quad S^2_X = \frac{n}{n-1} \sum_{k=1}^K f_k (X_k-\bar{X})^2 $$

\begin{tabularx}{0.8\textwidth}{p{1cm}p{1cm}YYYY}
\toprule
$X_k$ & $f_k$ & $f_k X_k$ & $(X_k-\bar{X})^2$ & $f_k(X_k-\bar{X})^2$\\
\midrule
1 & 0.7 & 0.7 & 0.09 & 0.063 \\
0 & 0.3 & 0 & 0.49 & 0.147 \\ \midrule
 & & 0.7 &  & 0.21 \\
\bottomrule \\
\end{tabularx}

\textit{Answer}: $$ \bar{X} = 0.7, \quad  S^2_X = \frac{100}{99} \cdot 0.21 = 0.21 $$

\newpage
\item You're in a statistics class with 30 students. Everyone takes the final exam, and the grades are all over the place. The average score for the class turns out to be 70, and the standard deviation is 10. You scored an 85. How did you fare relative to the class? \\~\\
Remember the Z-score formula:
$$ Z = \frac{X-\mu}{\sigma}   $$
How does your answer change if the standard deviation is 20? Why should the standard deviation affect your relative standing in the class? \\

When the standard deviation is 10:
$$ Z = \frac{85-70}{10} = 1.5   $$
You scored 1.5 standard deviations above the class average. \\

When the standard deviation is 20:
$$ Z = \frac{85-70}{20} = 0.75   $$
You are 0.75 standard deviations above the class average. \\

The standard deviation measures how spread out the grades are around the average. A smaller standard deviation (10 in this case) implies that most students scored close to the average. Your high score then stands out more, which is reflected in the higher Z-score (1.5).
\end{enumerate}

\end{document}