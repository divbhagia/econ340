\documentclass{./../../Latex/handout}
\begin{document}
\thispagestyle{plain}
\myheader{Handout for Lecture 18 \\ \normalsize{Omitted Variable Bias}}
\rhead{Class Handout for Lecture 18}
\vspace{-1em}

Consider the following linear regression model:
$$ Y = \beta_0 + \beta_1 X + u  $$
Omitted variable bias occurs when \underline{both} of the following conditions are true: 
\begin{itemize}
  \item[(1)] The omitted variable is correlated with $X$
  \item[(2)] The omitted variable is a determinant of $Y$
\end{itemize}

Consider the following model:
$$ Test Score = \beta_0 + \beta_1 \cdot STR + u $$
Here, \( Test Score \) represents the average test score on standardized tests for a school, and \( STR \) denotes the student-teacher ratio for that school.

\textit{Which of these omitted factors do you think will introduce bias in our OLS estimate for \( \beta_1 \)? Explain your reasoning.}
\begin{itemize}
  \item[(a)] percentage of English learners
  \item[(b)] time of day when tests were conducted
  \item[(c)] parking lot space per pupil (number of teacher parking spots divided by the number of students)
  \item[(d)] computers per student
\end{itemize}


\newpage
Omitted variable bias implies that the exogeneity assumption is violated and hence $E(u|X)\neq 0$. In which case, we have that:
$$ \hat{\beta}_1 = \beta_1 +  \frac{Cov(X,u)}{Var(X)} $$
So the direction and strength of bias $\hat{\beta}_1- \beta_1$ depends on the correlation between $u$ and $X$. 

In our example:
$$ Test Score = \beta_0 + \beta_1 \cdot STR + u $$

\textit{What should be the direction of bias due to the following omitted variables?}  
\begin{itemize}
  \item[(a)] percentage of English learners
  \item[(b)] computers per student
\end{itemize}

\end{document}