\documentclass{./../div_teaching_slides}
\usepackage{setspace}

\begin{document}
\title{ECON 340 \\ Economic Research Methods}
\author{Div Bhagia \\\vspace{1.75em}
Lecture 2 \\\vspace{0.25em} \small Empirical Distribution \& Measures of Central Tendency}
\date{}

\begin{frame}[noframenumbering, plain]
\maketitle
\end{frame}


%%%%%%%%%%%%%%%%%%%%
\begin{frame}{Describing Data}
A dataset is a collection of variables. Each variable contains multiple observations of the same measurement. \\~\\
\textit{Types of variables:} \\
\begin{witemize}
\item \textit{Categorical}: gender, race, education (\textit{binary}: two categories) 
\item \textit{Continuous}: income, age, GPA \\~\\
\end{witemize}
\textit{How do we summarize the information contained in a variable?} \\~\\
\end{frame}

\begin{frame}{The Empirical Distribution}
How often do different values occur? \\~\\
For categorical variables:
$$ f_k = \frac{n_k}{n}= \frac{\text{observations in category $k$}}{\text{total observations}} $$ \\~\\
$f_k$ captures the relative frequency of outcome $k$. 
\end{frame}

\begin{frame}{Frequency Distribution Table}
\setstretch{1.3}
% latex table generated in R 4.2.1 by xtable 1.8-4 package
% Thu Aug 24 10:46:35 2023
\begin{table}[ht]
\centering
\begin{tabular}{lll}
 Education & Count & Percent \\ 
  \hline
$<$ HS & 1540 &  6.39 \\ 
  HS Grad & 7388 & 30.64 \\ 
  Some College & 5595 & 23.20 \\ 
  4 Year College & 5979 & 24.80 \\ 
  $>$ College & 3611 & 14.98 \\ 
   \hline
Total & 24113 & 100 \\ 
  \end{tabular}
\end{table}

\end{frame}

\begin{frame}{Frequency Distribution Table}
\setstretch{1.3}
% latex table generated in R 4.2.1 by xtable 1.8-4 package
% Thu Aug 24 10:46:35 2023
\begin{table}[ht]
\centering
\begin{tabular}{llll}
 Education & Count & Percent & Cumulative \\ 
  \hline
$<$ HS & 1540 &  6.39 &   6.39 \\ 
  HS Grad & 7388 & 30.64 &  37.03 \\ 
  Some College & 5595 & 23.20 &  60.23 \\ 
  4 Year College & 5979 & 24.80 &  85.02 \\ 
  $>$ College & 3611 & 14.98 & 100.00 \\ 
   \hline
Total & 24113 & 100 &  \\ 
  \end{tabular}
\end{table}

\end{frame}

\begin{frame}{Histogram: Education}
\vspace{-0.5em}
\centering
\includegraphics{./../../Output/educ_hist.pdf}
\end{frame}

\begin{frame}{The Empirical Distribution}
What about continuous variables? \pause \\~\\
How often do different values occur in a particular interval? \\~\\
$$ f_k = \frac{\text{observations in \textit{interval} $k$}}{\text{total observations}} $$
\end{frame}

\begin{frame}{Histogram: Household Income}
\centering
\vspace{-0.9em}
\includegraphics{./../../Output/hhi_hist.pdf} \\
\scriptsize Source: American Community Survey (ACS) 2019
\end{frame}

%%%%%%%%%%%%%%%%%%%%
\begin{frame}{Measures of Central Tendency}
\begin{itemize}
\item[] \underline{\textbf{Mean}}: is the average value \\~\\ 
\item[] \underline{\textbf{Median}}: is the middle value \\~\\
\item[] \underline{\textbf{Mode}}: is the number that is repeated more often than any other \\~\\
\end{itemize} 
\underline{Example}: 5, 5, 10, 10, 10, 10, 20 
\end{frame}

%%%%%%%%%%%%%%%%%%%%
\begin{frame}{Mean}
\vspace{1em}
To calculate the mean:
$$ \bar{X} = \frac{\text{sum of all observations}}{\text{number of observations}} = \frac{1}{n}\sum_{i=1}^n X_i $$ \\~\\
Use $\bar{X}$ to denote the sample mean and $\mu$ to denote the population mean. 
\end{frame}


\begin{frame}{Mean vs Median}
\centering
\vspace{-0.75em}
\includegraphics{./../../Output/hhi_hist_mean_med.pdf} \\
\scriptsize Source: American Community Survey (ACS) 2019
\end{frame}

\begin{frame}{Mean vs Median}
\begin{witemize}
\item Mean household income: \$112,900
\item Median household income: \$91,600 \\~\\
\end{witemize} 
Why are mean earnings higher than the median?
\end{frame}

\begin{frame}{Percentiles}
The \textbf{$P^{th}$ percentile} is a value such that $P$\% of observations are at or below that number. \\~\\
\begin{itemize}
\item [] 25th percentile a.k.a 1st quartile
\item [] 75th percentile a.k.a 3rd quartile \\~\\
\end{itemize}
\textit{What is the 50th percentile called?} \\~\\
\end{frame}

\begin{frame}{More about Mean}
\vspace{0.5em}
\begin{itemize}
\item $\sum_{i=1}^n X_i = n \bar{X}$ \\~\\ \pause
\item Deviations from the mean are always zero
$$ \sum_{i=1}^n (X_i-\bar{X}) =  \sum_{i=1}^n X_i - n \bar{X} = n \bar{X}- n \bar{X}=0 $$ \pause
\item We can always write
$$ \bar{X} = \frac {\sum_{i=1}^n X_i}{n} = \frac{1}{n}\sum_{i=1}^n X_i = \sum_{i=1}^n \frac{X_i}{n} $$
\end{itemize}
\end{frame}

\begin{frame}[t]{An easier way to calculate mean}
\vspace{-0.5em}
\begin{witemize}
  \item If data is grouped, we can use the frequency distribution table to calculate the mean: \\ 
$$ \bar{X} = \frac{\sum_{k=1}^K n_k X_k}{n} = \sum_{k=1}^K f_k X_k  $$ \\~\\ \vspace{-0.5em}
  \item Previous example: 5, 5, 10, 10, 10, 10, 20 \\
\end{witemize}
\begin{center}
\begin{tabular}{|c|c|p{1cm}|p{2cm}|}
\hline
$X_k$ & \hspace{0.5em} $n_k$ & \hspace{0.2em} $f_k$ & \hspace{0.2em} $X_k f_k$ \\
\hline
 5 & 2 & & \\
 \hline
10 &  4 & & \\
\hline
20 &  1 & & \\
\hline
Total & 7  & &  \\
\hline 
\end{tabular} 
\end{center}
\vspace{0.5em}

\end{frame}

\begin{frame}{Weighted Mean}
The weighted mean of a set of data is 
$$ \bar{X} = \frac{\sum_{i=1}^nw_i X_i}{\sum_{i=1}^n w_i} $$
where $w_i$ is the weight of the $i^{th}$ observation. \\~\\

Why might we want to use a weighted mean?
\end{frame}

\begin{frame}{2016 Election Predictions}
\centering
 \includegraphics[scale=0.14]{nyt2.png} \\
\includegraphics[scale=0.19]{nyt.png} \\
\end{frame}


\begin{frame}{Things to do next}
\begin{witemize}
\item Review this week's material; handouts and reading (NYT article) on Canvas
\item You may be asked to summarize what you got out of the reading in the next class
\end{witemize}
\end{frame}

\end{document}