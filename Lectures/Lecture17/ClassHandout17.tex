\documentclass{./../../Latex/handout}
\begin{document}
\thispagestyle{plain}
\myheader{Handout for Lecture 17 \\ \normalsize{Inference in Regression Models}}
\rhead{Class Handout for Lecture 17}
\vspace{-1em}

We are interested in the following relationship:
$$ Crime Rate = \beta_0 + \beta_1 Unemployment Rate + u  $$
We run a regression of \textit{Crime Rate} on \textit{Unemployment Rate} across US counties. The output from this regression is presented below. \footnote{Crime rate is defined as the ratio of crimes in a specific area to the population of that area, expressed as incidents per 1,000 people per year. The unemployment rate represents the number of unemployed people as a percentage of the labor force. The average crime rate in the data is 0.0057, while the average unemployment rate is 0.05.}

\centering

% Table created by stargazer v.5.2.3 by Marek Hlavac, Social Policy Institute. E-mail: marek.hlavac at gmail.com
% Date and time: Thu, Oct 26, 2023 - 13:53:33
\begin{tabular}{@{\extracolsep{5pt}}lc} 
\\[-1.8ex]\hline 
\hline \\[-1.8ex] 
\\[-1.8ex] & Crime Rate \\ 
\hline \\[-1.8ex] 
 Intercept & 0.003$^{***}$ \\ 
  & (0.0002) \\ 
  & \\ 
 Unemployment Rate & 0.05$^{***}$ \\ 
  & (0.004) \\ 
  & \\ 
\hline \\[-1.8ex] 
Observations & 2,957 \\ 
R$^{2}$ & 0.05 \\ 
\hline 
\hline \\[-1.8ex] 
\textit{Note:}  & \multicolumn{1}{r}{$^{*}$p$<$0.1; $^{**}$p$<$0.05; $^{***}$p$<$0.01} \\ 
\end{tabular} 
 
\begin{enumerate}
	\item Interpret the slope and intercept coefficient.
	\item Interpret the $R^2$.
	\item According to this model, what is the predicted Crime Rate for a county with an unemployment rate of 0.06 (6\%)? 
	\item What is the t-statistic associated with the hypothesis test that examines whether the coefficient on the unemployment rate is zero ($H_0: \beta_1 = 0$)?
	\item What is the smallest level of significance at which you would reject the hypothesis in question 4? 
	\item Construct a 95\% confidence interval for $\beta_1$. (Note: $Pr(|Z|>1.96)=0.95$.) What is the interpretation of this interval?
	\item Can we attach a causal interpretation to the estimates here? Why or why not?
\end{enumerate}


\end{document}