\documentclass{./../../Latex/handout}

\fancyhf{}
\fancyheadoffset{1cm}
\lhead{ECON 340: Economic Research Methods}
\rhead{Div Bhagia}
\rfoot{\center \thepage}

\begin{document}
\thispagestyle{plain}
\begin{center}
  \framebox{
    \vbox{
    \vspace{0.1em}
     \Large \sc Handout for Lecture 2 \\ \vspace{0.1em}
     \large Empirical Distribution and Measures of Central Tendency \\
     \vspace{0.5em}
      \small ECON 340: Economic Research Methods \hfill Instructor: Div Bhagia
      }
    }
\end{center}	

\begin{enumerate}
\item You rolled a six-sided die 100 times and noted down how many times each of the six outcomes was realized. Fill in the rest of the table below: \\~\\
\setstretch{1.5}
\begin{tabular}{c|c|c|c}
Outcome & Count ($n_k$) & Relative frequency ($f_k$) & Cumulative frequency ($F_k$) \\
\hline
1 &  18 & 0.18 & 0.18 \\
\hline
2 & 18 & 0.18 & 0.36 \\
\hline
3 & 12 & 0.12 & 0.48 \\
\hline
4 & 16 & 0.16 & 0.64 \\
\hline
5 & 21 & 0.21 & 0.85 \\
\hline
6 & 15 & 0.15 & 1 \\
\hline
\textbf{Total} & \textbf{100} & \textbf{1} & \\
\hline
\end{tabular}
\vspace{0.5em}

Note that $$ f_k = \frac{n_k}{n} = \frac{\text{observations in category $k$}}{\text{total observations}} $$

\begin{enumerate}
	\item How many times did you get a die face with a value of at most 3? 48 \\ 
	\item Are the proportions close to what you would have predicted? 
\item[] Yes, we would have predicted each outcome's frequency to be close to 1/6 (or 0.16).
\end{enumerate}

\item Find the mean and median for: $ 3, 4, 1, 6, 8 $
$$ mean = \frac{3+4+1+6+9}{5} = \frac{22}{5} = 4.4 $$
Arrange in ascending order, then the middle number is the median: $ 1, 3, \mathbf{4}, 6, 8$.

\item Amongst the mean and the median, which one is more affected by outliers? Explain. 
\item[] The mean is more sensitive to outliers because all values are used in its calculation, unlike the median, which is the middle value and less affected by extreme values.
\item We asked a sample of 10 individuals whether they like icecream or not. We then created a variable $X$ that takes value 1 if the individual likes icecream, and 0 otherwise.  Here is the data we collected:
$$ 1, 1, 0, 0, 0, 1, 0, 1, 1, 1 $$

\begin{enumerate}
\item How many individuals like icecream in our sample? 6
\item What proportion of individuals like icecream in our sample? 6/10 = 0.6
\item Use the frequency distribution table and the following formula to calculate the mean of $X$. 
$$ \bar{X} = \frac{\sum_{k=1}^K n_k X_k}{n} = \sum_{k=1}^K f_k X_k = 0.6 $$

\begin{center}
\begin{tabularx}{0.5\textwidth}{Y|Y|Y|Y}
  $X_k$ & $n_k$ & $f_k$ & $X_k f_k$ \\
  \hline
  1 & 6 & 0.6 & 0.6 \\
  0 & 4 & 0.4 & 0 \\
\hline
& & &  0.6
\end{tabularx}
\end{center}


\end{enumerate}

\newpage
\item We have the following data on shoe sizes ($X_i$) for four individuals. 
$$ X = \{8, 6, 6, 8 \}$$
\begin{enumerate}
\item Calculate the mean:
$$ \mu = \frac {\sum_{i=1}^N X_i}{N} = \frac{8 + 6 + 6 + 8}{4} = \frac{28}{4} = 7 $$ 
\item Calculate the weighted mean with weights $ w=  \{1, 1, 1, 1 \}$.
$$ \mu_{Weighted} = \frac {\sum_{i=1}^N w_i X_i}{\sum_{i=1}^N w_i } = \frac{1\times 8 + 1\times 6 + 1\times 6 + 1\times 8}{1+1+1+1} = 7 $$ 
\item Calculate the weighted mean with weights $ w=  \{1, 2, 2, 1 \}$.
$$ \mu_{Weighted} = \frac {\sum_{i=1}^N w_i X_i}{\sum_{i=1}^N w_i } = \frac{1\times 8 + 2\times 6 + 2\times 6 + 1\times 8}{1+2+2+1} = 6.66  $$ 
\item Calculate the weighted mean with weights $ w=  \{0.5, 0, 0, 0.5 \}$.
$$ \mu_{Weighted} = \frac {\sum_{i=1}^N w_i X_i}{\sum_{i=1}^N w_i } = \frac{0.5\times 8 + 0\times 6 + 0\times 6 + 0.5\times 8}{0.5 + 0 + 0 + 0.5} =8  $$ 
\end{enumerate}
\vspace{4em}
\end{enumerate}

\end{document}