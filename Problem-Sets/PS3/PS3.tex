\documentclass{./../../Latex/handout}
\begin{document}
\thispagestyle{plain}
\myheader{Problem Set 3}
\rhead{Problem Set 3}

\textit{Please make sure to show all your work for each question, and include the formulas used for your calculations when applicable.}

\begin{enumerate}

\item (3 pts) Suppose you are given that $X$ follows a normal distribution with mean 20 and variance 49 i.e. $X \sim N(20,49)$. 
 \begin{enumerate}
 \item What is the probability that $X$ is equal to 15? \vspace{1.25cm}
\item What is the probability that $X$ is less than 15? \vspace{2cm}
\item What is the probability that $X$ is between 15 and 25? \vspace{3cm}
\item The probability that $X$ is more than $x_0$ is 0.8. What is $x_0$? \vspace{3cm}
\item Your friend is surprised that you were quickly able to solve complex integrals to answer these questions. Can you explain to your friend how you did it? \\~\\
\end{enumerate} 

\newpage
\item (3 pts) The table below gives us the joint probability distribution of time taken to read five pages of a book ($Y$)  and caffeine consumption ($X$). \\~\\
\begin{tabularx}{0.95\textwidth}{cccc}
\toprule
	& Caffeine $(X=1)$ & No Caffeine $(X=0)$ & Total \\
	\midrule
10-mins ($Y=10$) & 0.2  & 0.1 & 0.3 \\ 
20-mins ($Y=20$) & 0.3 & 0.4 & 0.7 \\
\midrule
 Total & 0.5 & 0.5 & 1 \\
 \bottomrule \\
\end{tabularx}
\begin{enumerate}
\item Find $E(Y)$, which is the (unconditional) expected value of time taken to read five pages. \vspace{2.5cm}
\item Now calculate $E(Y|X=1)$ and $E(Y|X=0)$. \vspace{4.5cm}
\item Given your answer in (b), are $X$ and $Y$ independent variables? Explain. \vspace{3.25cm}
\item What can you say about the impact of caffeine on reading speed from your answer in (b)? \vspace{2cm}
\end{enumerate}

\item (2 pts) We have reasons to believe that the distribution of household income
for the US population is right-skewed. I am thinking about
taking a random sample of 10,000 individuals from the US population
and calculating the mean.
\begin{enumerate}
  \item Is it possible to ascertain whether the distribution of the sample mean in this case will be normal? If yes, how did you come to this conclusion? \vspace{3.25cm}
\item Say the true population mean of income is \$90,000 and the true standard
deviation is \$68,000. What is the expectation and the variance of the sample mean in this case? \vspace{3.25cm}
\end{enumerate}

\item (2 pts) We took a random sample of 25 CSUF students and asked them how many hours did they spend in front of a screen in the past one week. Say we know that hours in front of the screen for CSUF students is normally distributed with unknown mean and variance $\sigma^2 = 36$.  We found that average hours in front of a screen in our sample was 40. Construct a 95 percent confidence interval for the population mean. \\

\end{enumerate}
\end{document}