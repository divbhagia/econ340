\documentclass{./../../Latex/handout}
\begin{document}
\thispagestyle{plain}
\myheader{Problem Set 1 Solutions}
\rhead{Problem Set 1 Solutions}


\begin{enumerate}
\setlength\itemsep{2em}

\item (1 pt) The statement is \underline{true} because 
\begin{align*} \sum_{i=1}^n (3X_i^2 + 2k) &=  \sum_{i=1}^n 3 X_i^2 + 2nk \\
&= 3 \sum_{i=1}^n  X_i^2 + 2nk  \\
&= 3 \sum_{j=1}^{n-1} X^2_j + 3X^2_n  + 2nk  &
\end{align*} 
The last step follows from the fact that 
$$ \sum_{i=1}^n X^2_i = \underbrace{X^2_1 + X^2_2 + ... + X^2_{n-1}}_{\sum_{i=1}^{n-1} X^2_i} + X^2_n $$
So we can re-write it as 
$$  \sum_{i=1}^n X^2_i  = \sum_{i=1}^{n-1} X_i^2 + X_n^2 $$ 
Also, since $i$ and $j$ are just indexes for each observation, it doesn't matter what letter we use. 

\item (1 pt) Fill in the following frequency distribution table. 
\begin{center}
\begin{tabular}{|c|c|c|}
\hline
$X_i$ & Relative & Cumulative \\
& Frequency & Frequency \\
\hline
1 & \textbf{0.1} & 0.1 \\
\hline
2 & \textbf{0.2}  & 0.3 \\
\hline
3 & 0.4 & \textbf{0.7}   \\
\hline
4 & \textbf{0.3} & \textbf{1} \\
\hline
Total & \textbf{1}  & X \\
\hline
\end{tabular}
\end{center}

\item (2 pts) Data on calorie consumption of five adults is given:
$$ 1400, 1800, 2000, 2200, 2600 $$
\begin{enumerate}
\item The mean calorie consumption is given by 
\begin{align*}
 \mu  = \frac{\sum_{i=1}^5 X_i}{5}  & = \frac{1400+1800+2000+2200+2600}{5} \\
 & = \frac{10000}{5} = 2000 
 \end{align*}
\item Since $n=5$ is odd, the median calorie consumption for this group is given by 
$ \frac{n+1}{2}$ th term which is 2000. 
\end{enumerate}
\textit{We added one more individual to the data set whose calorie consumption is $4100$.}
\begin{enumerate}
\item[(c)] Since we added another individual to the data 
$$  \mu_{new} = \frac{\sum_{i=1}^6 X_i}{6} = \frac{10000+4100}{6} = \frac{14100}{6} = 2350 $$
\textit{Aside:} If we were not given the calorie consumption of the first five individuals but just their mean $\mu_{old} = 2000$, we could still find the new mean with the addition of one more person. How?
$$ \mu_{new} = \frac{\sum_{i=1}^6 X_i}{6} = \frac{\sum_{i=1}^5 X_i + 4100}{6} = \frac{5 \mu_{old}  + 4100}{6}$$ 
\item[(d)]  Now $n=6$ is even so the median calorie consumption is average of $\frac{n}{2}$th and $\frac{n}{2}+1$th term. So the median is 
$$ \frac{2000+2200}{2} = 2100 $$
\item[(e)] Mean is more susceptible to outliers than the median. This is because every observation contributes to the mean with a weight of $\frac{1}{n}$. So if an observation is disproportionately larger (smaller) than others, it can pull the mean upwards (downwards). Whereas the median only depends on observations in the center of the data. 
\end{enumerate}

\item (1 pt) Here is the amount (in \$) that I spent on groceries in the last three weeks: 
$$ 100, 120, 80  $$
To calculate the variance, we first need the average spending on groceries in the last three weeks:
$$ \mu = \frac{\Sigma_{i=1}^3 X_i}{3} = \frac{100+120+80}{3} = 100 $$
Now to calculate the variance:
\begin{align*}
	\sigma^2 &= \frac{\Sigma_{i=1}^3 (X_i-\mu)^2}{3} \\
	&= \frac{(100-100)^2+(120-100)^2+(80-100)^2}{3} \\ 
	&= \frac{0+400+400}{3} = 266.67
\end{align*}

\item (3 pts) The following table is constructed from a \underline{sample} of 6 students. $X_i$ represents the number of hours an individual usually sleeps and $Y_i$ represents the number of hours the individual typically exercises per week.  
\begin{center}
\footnotesize
\def\arraystretch{1.5}% 
\begin{tabular}{|c|c|c|c|c|c|c|c|}
\hline
Obs & $X_i$ & $Y_i$ & $(X_i-\bar{X})$ & $(Y_i-\bar{Y})$ & $(X_i-\bar{X})^2$ & $(Y_i-\bar{Y})^2$ & $(X_i-\bar{X})(Y_i-\bar{Y})$  \\ \hline
1 & 8 & 3 & 0.5 & -1 & 0.25 & 1 & -0.5 \\ \hline
2 & 7 & 4 & -0.5 & 0 & 0.25 & 0 & 0 \\ \hline
3 & 6.5 & 2 & -1 & -2 & 1 & 4 & 2 \\ \hline
4 & 7.5 & 4 & 0 & 0 & 0 & 0 & 0 \\ \hline
5 & 9 & 6 & 1.5 & 2 & 2.25 & 4 & 3 \\ \hline
6 & 7 & 5 & -0.5 & 1 & 0.25 & 1 & -0.5 \\ \hline
\textbf{Total} & \textbf{45} & \textbf{24} & \textbf{0} & \textbf{0} & \textbf{4} & \textbf{10} & \textbf{4} \\ \hline
\end{tabular}
\end{center}
\begin{enumerate}
\item What is the variance of $X$ and $Y$? 
\item [] We can use the formula for the sample variance:
$$ S_X^2 = \frac{1}{n-1} \sum_{i=1}^n (X_i-\bar{X})^2 $$
From the above table, we already have $\sum_{i=1}^n (X_i-\bar{X})^2 = 4$. Here $n=6$, so $n-1=5$. Plugging this in the formula we get: 
$$ S_X^2 = \frac{4}{5} = 0.8 \quad \quad  S_Y^2 = \frac{10}{5} = 2 $$ \\
\item What is the standard deviation of $X$ and $Y$? 
\item [] Standard deviation is given by the square root of the variance. So we have: $$ S_X = \sqrt{S^2_X} = \sqrt{0.8} = 0.89 $$
 $$ S_Y = \sqrt{S^2_Y} = \sqrt{2} = 1.41 $$
\item How many standard deviations is the fifth observation away from the average hours of sleep? 
\item [] The average hours of sleep is given by:
 $$ \bar{X} = \frac{\sum_{i=1}^n X_i}{n} = \frac{45}{6} = 7.5 $$  The fifth observation is $9-7.5=1.5$ hours away from the average. The standard deviation for hours of sleep is 0.89. So the fifth observation is $1.5/0.89 = 1.68$ standard deviations away from the mean. \\
\item What is the covariance between hours of sleep per night and hours of exercise per week? 
\item [] Formula for sample covariance:
\item[] $$ S_{XY} = \frac{1}{n-1}\sum_{i=1}^n (X_i-\bar{X})(Y_i-\bar{Y}) $$
Since we have $\sum_{i=1}^n (X_i-\bar{X})(Y_i-\bar{Y})=4$ from the table. $$ S_{XY} = \frac{4}{5} = 0.8$$
\item What is the correlation between $X$ and $Y$? \\~\\
 $$ r_{XY} = \frac{S_{XY}}{S_X S_Y} = \frac{0.8}{0.89 \times 1.41} = 0.64 $$ \\
\end{enumerate}

\item (2 pts) For this question, we can use the formulas to calculate mean and variance for grouped data.
\begin{center}
\setstretch{2}
\begin{tabularx}{0.9\textwidth}{|@{}Y| @{}Y |@{}Y| @{}Y| @{}Y| @{}Y@{}|}
\hline
 $X_k$ & $n_k$ & $f_k$ & $f_k X_k$ & $(X_k-\bar{X})^2$ & $f_k (X_k-\bar{X})^2$ \\  \hline
 1 & 200 & 0.2 & 0.2 & 0.64 & 0.128\\ \hline
 0 & 800 & 0.8 & 0 &  0.04 & 0.032 \\ \hline
Total & 1000 & 1 & 0.2 & NA & 0.16 \\ 
\hline
\end{tabularx}	
\end{center}
So we can calculate the mean and variance of $X$ as follows:
$$\bar{X} = \sum_{k=1}^K f_k X_k = 0.2  $$
$$ S_X^2 = \frac{n}{n-1} \sum_{k=1}^K f_k (X_k-\bar{X})^2 = \frac{1000}{999}\cdot 0.16 \approx 0.16 $$ 

The variance without sample correction would be given by:
$$ \sigma_X^2 = \sum_{k=1}^K f_k (X_k-\mu)^2 =  0.16 $$ 

\end{enumerate}
\end{document}