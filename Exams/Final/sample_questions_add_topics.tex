\documentclass{./../../Latex/handout}
\usepackage{tasks}
\begin{document}
\thispagestyle{plain}
\myheader{Sample Questions for the Additional Topics Module}
\rhead{Sample Questions}
\vspace{-1em}

You can expect open-ended or multiple-choice questions derived from the following topics on the final exam. \\

\textbf{Randomized Controlled Trials} \vspace{-1em}
\begin{enumerate}
\item What is a Randomized Controlled Trial (RCT)?
\item Why is evidence from RCTs superior to evidence from observational data?
\item How does randomization in RCTs contribute to the validity of the study?
\item Can you provide examples of situations where a RCT would be infeasible or unethical?
\end{enumerate}

\textbf{Quasi-experimental Methods} \vspace{-1em}
\begin{enumerate}
\item Can you provide a brief explanation of a commonly used method in
economics to evaluate policies when RCTs are not
feasible or ethical?
\item Consider a scenario where a government policy provides additional educational funding to schools if their pass rate in standardized exams is above 75\%. Those falling below this threshold don't receive the funds. Given this clear cutoff point, how would you design a study to evaluate the impact of this funding on school performance?
\item Explain how a Regression Discontinuity (RD) design can be used to assess the impact of a policy intervention. 
\item Considering we have both pre- and post-implementation data for all US states, what would be the most effective approach to evaluate the impact of a policy that has been implemented in some states but not in others?
\item Describe how the Difference-in-Differences (DID) approach is utilized to evaluate the effects of a policy change. 
\item Let \( Y^T_0 \) denote the average pre-treatment outcome for the treated group and \( Y^C_0 \) for the control group. Similarly, let \( Y^T_1 \) and \( Y^C_1 \) represent the average post-treatment outcomes for the treated and control groups, respectively. Write the formula for the Difference-in-Differences (DiD) estimator based on these variables.
\item What is the key assumption underlying the Difference-in-Differences (DiD) methodology when evaluating the effects of a policy change?
\item Discuss the parallel trends assumption that is critical to the Difference-in-Differences methodology. Why is it important to ensure that the treatment and control groups exhibit parallel trends in outcomes before the policy intervention?
\end{enumerate}

\textbf{Big Data \& Machine Learning} \vspace{-1em}
\begin{enumerate}
\item How does Machine Learning differ from traditional Econometrics?
\item Explain the relationship between Machine Learning and Artificial Intelligence.
\end{enumerate}


\end{document}

\end{document}