\documentclass{./../../Latex/handout}
\begin{document}
\thispagestyle{plain}
\myheader{Final Exam Study Guide}
\rhead{Final Exam Study Guide}

\textit{Here is some general information about the final exam.}
\begin{itemize}
\item This will be a closed-book exam.
\item You will be allowed to use a calculator. No phones or computers are allowed. 
\item It is really important that you review all the material uploaded on the Course Website under the modules \textit{Linear Regression Model} and \textit{Additional Topics}.
\item You do not need to know any proofs for the exam.

\item While writing the exam, make sure to show your work to get full credit. This also helps you get partial credit in case you make a small error.
\end{itemize}

 \textit{What do you need to know?}
\begin{enumerate}
\item How does the Ordinary Least Squares (OLS) method estimate the coefficients of a regression model? 
\item OLS line passes through sample means. 
\item Given the estimated model, 
$$ \hat{Y}_i = \hat{\beta_0} + \hat{\beta_1} X_i $$

If I tell you the value of $X_j$ and $Y_j$ for observation $j$, you should be able to calculate the prediction ($\hat{Y}_j$) and the error term ($\hat{u}_j$) for this observation. (Also, be able to do so for the multiple regression model).
\item Understand the formula for $R^2$ and how to interpret it. When is $R^2=1$ or $R^2=0$? 
\item Four assumptions for the simple linear regression model. 
\item Difference between causal and predictive analysis. 
\item Multiple regression: interpretation, adjusted-$R^2$
\item Be able to interpret the output from the following regression models:
\begin{itemize}
  \item Linear
  \item Quadratic
  \item Log-Log (elasticity), Log-Level, Level-Log
  \item Dummy variable 
  \item Interaction terms
\end{itemize}
For example, if I give you the following model:
$$ \log Y = \beta_0 + \beta_1 X $$
You should be able to take the derivative and show:
$$ \frac{1}{Y} \cdot \frac{dY}{dX} = \beta_1 \rightarrow 100 \beta_1 = \frac{100 \times dY/Y}{dX}  $$
So we can interpret $100 \beta_1 $ as the percentage change in $Y$ in response to a one unit change in $X$.
\item Given the coefficient estimate and its standard error, be able to set up a hypothesis test to check statistical significance. Calculate and interpret $p$-values for regression coefficients. Construct confidence intervals. 
\item Be able to identify the direction of omitted variable bias. 
\item What are randomized control trials? What are some tools economists use to emulate experimental variation?
\item What are the differences between Machine Learning (ML) and Econometrics? Are ML methods geared more towards predictive or causal inference?
\end{enumerate}

\end{document}
