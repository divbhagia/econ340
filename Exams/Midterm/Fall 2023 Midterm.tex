\documentclass{./../../Latex/tests}
\usepackage{multirow} % Add to class

\begin{document}
\thispagestyle{plain}
\myheader{Fall 2023 Midterm}
\rhead{Fall 2023 Midterm}
\vspace{0.5em}
\testHeader{70}{20}


%%%%%%%%%%%% Question 1
\underline{Question 1: Multiple Choice Questions (1 pt each, total 5 pts)}

Choose a single correct response for all questions.
\begin{enumerate}
	
% Part (a)	
\item[(a)] What does the variance of the sample mean signify?
\begin{enumerate}
\item[$\square$] Dispersion in sample data
\item[$\square$] Sample-to-sample variability
\item[$\square$] Spread of the population data
\item[$\square$] Bias in sample mean \\
\end{enumerate}

% Part (b)	
\item[(b)] We are choosing between two estimators $\hat{\theta}_1$ and $\hat{\theta}_2$ for some population parameter $\theta$, both of which are unbiased i.e. $E(\hat{\theta}_1)=\theta$ and $E(\hat{\theta}_2)=\theta$. But the variance of $\hat{\theta}_1$ is lower than that of $\hat{\theta}_2$ i.e. $Var(\hat{\theta}_1)<Var(\hat{\theta}_2)$.  Which of the following is true? 
\begin{itemize}
\item[$\square$] We are indifferent between the two estimators.
\item[$\square$] We prefer $\hat{\theta}_1$ over $\hat{\theta}_2$.
\item[$\square$] We prefer $\hat{\theta}_2$ over $\hat{\theta}_1$.
\item[$\square$] We need more information to reach any conclusion. \\~\\
\end{itemize}

% Part (c)	
\item[(c)] If the average score on a math test is 75 with a standard deviation of 10, and your score is 70, then you are:
\begin{enumerate}
\item[$\square$] 1/2 standard deviation over the mean
\item[$\square$] 1 standard deviation over the mean
\item[$\square$] 1/2 standard deviation below the mean
\item[$\square$] 1 standard deviation below the mean \\
\end{enumerate}

\newpage
% Part (d)	
\item[(d)] In a scientific study, researchers are analyzing the correlation between the variable \( X \), which represents the age of trees, and \( Y \), which measures the height of those trees in centimeters. The researchers also create a new variable \( Z \) that measures the height of the trees in inches. Which of the following is true?

\begin{itemize}
\item[$\square$] \( \text{cor}(X, Y) >\text{cor}(X, Z) \)
\item[$\square$] \( \text{cor}(X, Y) < \text{cor}(X, Z) \)
\item[$\square$] \( \text{cor}(X, Y) = \text{cor}(X, Z) \)
\item[$\square$] None of the above \\
\end{itemize}


% Part (e)	
\item[(e)] Which of the following statements correctly explains the concept of $p$-values in hypothesis testing?
\begin{enumerate}
\item[$\square$] The $p$-value is the probability that the null hypothesis is true.
\item[$\square$] The $p$-value is the probability of observing the null hypothesis given the sample data.
\item[$\square$] The $p$-value is the probability of observing the sample data or more extreme values, assuming the null hypothesis is true.
\item[$\square$] The $p$-value is the probability of observing a statistically significant result in a hypothesis test.
\end{enumerate}
\end{enumerate}


\newpage
%%%%%%%%%%%% Question 2
\underline{Question 2: Things you can explain. (5 pts)}
\begin{enumerate}

% Part (a)	
\item[(a)] (2.5 pts) If a study discovers a strong correlation between education level and wealth, can we infer that higher education leads to increased wealth?
\vspace{7.5cm}

% Part (b)	
\item[(b)] (2.5 pts) Imagine a university reports that the average GPA of its student body is 3.4, based on university-wide data. A student organization surveys a random sample of 100 students and finds that the average GPA is 3.2. Should the student organization be concerned that they found a different average GPA than the university? Explain your reasoning, considering factors such as sampling variation and sample size.
\end{enumerate}


%%%%%%%%%%%% Question 3
\newpage
\underline{Question 3: Investor Behavior (6 pts)}

Imagine you're a financial advisor studying two types of investors: \textit{risk-averse} (denoted by $X=1$) and \textit{risk-taking} (denoted by $X=2$). You're interested in the amount of money $Y$ they are willing to invest in a new startup: either \$10,000, \$20,000, or \$30,000.

The joint probability distribution table below shows the likelihood of an investor from each risk category investing a particular amount in a new startup: \\

\begin{tabularx}{\textwidth}{|p{4cm}|Y|Y|Y|Y|}
\hline
	& \multicolumn{3}{c|}{Investment Amount ($Y$)} & \multirow{2}{*}{Total} \\
	\cline{2-4} 
	& \$10,000 & \$20,000 & \$30,000 &  \\ 
	\hline
 Risk-averse $(X =1)$	& 0.35 & 0.15 & 0.10 & \\  
 \hline
 Risk-taking $(X =2)$ 	& 0.05 & 0.15 & 0.20 & \\
\hline
Total & & & &\\
\hline
\end{tabularx} \\

Answer the following questions. \textit{For each question write down the formula you are using if any and show your work clearly. }

\begin{enumerate}
% Part (a)
\item[(a)] (1 pt) What proportion of all investors invest \$10,000?
\vspace{2cm}

% Part (b)
\item[(b)] (1 pt) What is the expected amount of investment across all investors?
\vspace{5.75cm}

% Part (c)	
\item[(c)] (1 pt) What proportion of investors are risk-averse?
\vspace{1.5cm}

\newpage
% Part (d)	
\item[(d)] (1 pt) What \textit{proportion} of risk-averse investors invest \$10,000?
\vspace{2cm}

% Part (e)	
\item[(e)] (1 pt) What is the expected amount of investment from a risk-averse investor?
\vspace{10cm}

  
% Part (d)	
\item[(f)] (1 pt) Given your answers above, do you think the investment amount is independent of an individual's risk appetite?

\end{enumerate}

%%%%%%%%%%%% Question 4
\newpage
\underline{Question 4: Estimating Average Income (4 pts)}

In an economics study, researchers are interested in estimating the average monthly income of households in a particular city. They take a random sample of 150 households and find the average monthly income to be \$4,500 with a standard deviation of \$950.

\begin{enumerate}

% Part (a)
\item[(a)] (1.5 pts) Construct a 95\% confidence interval for the average monthly income. (Note that: $Pr(Z>1.96) = 0.025$.)
\vspace{4.5cm}

% Part (b)
\item[(b)] (1 pt)  Interpret the 95\% confidence interval you've constructed in part (a). What does it mean in the context of the study?
\vspace{4.5cm}

% Part (c)
\item[(c)] (1.5 pts) If the researchers wanted to decrease the width of the confidence interval while keeping the confidence level constant, what could they do? 
\end{enumerate}

\end{document}