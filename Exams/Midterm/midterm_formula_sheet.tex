\documentclass{./../../Latex/handout}
\rhead{Midterm Formula Sheet}
\begin{document}

\underline{Descriptive Statistics} \\~\\
Let $n$ denote the number of sample observations and $N$ denote the number of population observations.
\begin{center}
\setstretch{1.75}
\begin{tabular}{|l|l|l|}
\hline
  Statistic  & Sample Formula & Population Formula \\
    \hline
    Mean & $ \bar{X} = \frac{1}{n} \sum_{i=1}^n X_i $  & $ \mu_X = \frac{1}{N} \sum_{i=1}^N X_i $ \\ \hline
    Variance & $ S_X^2 = \frac{1}{n-1} \sum_{i=1}^n (X_i-\bar{X})^2 $ & $ \sigma_X^2 = \frac{1}{N} \sum_{i=1}^N (X_i-\mu_X)^2 $
  \\ \hline 
St. Dev. & $\sqrt{S_X^2}$ & $\sqrt{\sigma_X^2}$ \\ \hline
Covariance & $S_{XY} = \frac{1}{n-1}\sum_{i=1}^n (X_i-\bar{X})(Y_i-\bar{Y})$ & $ \sigma_{XY} = \frac{1}{N}\sum_{i=1}^N (X_i-\mu_X)(Y_i-\mu_Y)$ \\ \hline
Correlation & $ r_{XY} = \frac{ S_{XY}}{S_X S_Y}$ & $\rho_{XY} = \frac{ \sigma_{XY}}{\sigma_X \sigma_Y} $ \\ \hline 
\end{tabular}	
\end{center} 

\vspace{1em}
\hrule 
\vspace{0.5em}
\underline{Frequency and Statistics from Grouped Data} \\~\\
Let $n$ denote the total number of observations and $n_k$ denote the number of observations in category $k$, relative frequency $f_k = n_k/n$. There are $J$ groups or bins.

\begin{center}
\setstretch{1.75}
\begin{tabular}{|l|l|l|}
\hline
  Statistic  & Sample Formula & Population Formula \\
    \hline
    Mean & $ \bar{X} = \sum_{k=1}^J f_k X_k $  & $ \mu = \sum_{k=1}^J f_k X_k $ \\ \hline
    Variance & $ S_X^2 = \frac{n}{n-1} \sum_{k=J}^J f_k (X_k-\bar{X})^2  $ & $ \sum_{k=1}^J f_k (X_k-\mu_X)^2  $
  \\ \hline 
\end{tabular}	
\end{center} 
\vspace{0.5em}

\hrule 
\vspace{0.5em}
\underline{Weighted Mean} 
$$ \bar{X}^{\omega} = \frac{\sum_{i=1}^n\omega_i X_i}{\sum_{i=1}^n \omega_i} $$
where $\omega_i$ is the weight of the $i^{th}$ observation. \\

\hrule \vspace{0.5em}
\underline{Z-Score}
$$ Z = \frac{X - \mu}{\sigma} $$ 
\hrule \vspace{0.5em}
\newpage
\underline{Single Random Variable (RV)} 
\begin{center}
\setstretch{1.75}
\begin{tabular}{|l|l|l|}
\hline
  Quantity  & Discrete RV & Continuous RV \\
    \hline
    Probability/frequency & $  f(x) = Pr(X=x)$  & $ Pr(a \leq X \leq b) =  \int_{a}^{b} f(x) \partial x $ \\ \hline
    Cumulative prob., $Pr(X \leq x_0)$ & $\sum_{x \leq x_0} f(x)$ & $ \int^{x_0}_{-\infty} f(x)\partial x $ \\ \hline
    Expected value, $E(X)$ or $\mu_X$ & $\sum_x x f(x)$ & $\int_x x f(x)$ dx
  \\ \hline 
\end{tabular}	
\end{center} 
$$Var(X) = \sigma_X^2 = E [(X-\mu)^2]   $$ 

\hrule \vspace{0.5em}
\underline{Multiple Random Variables} 

Conditional probability (discrete): $$ f(y | x) = Pr(Y=y| X=x) = \frac{Pr(X=x, Y=y)}{Pr(X=x)} = \frac{f(x,y)}{f(x)}$$	
Conditional expectation (discrete):
$$ E(Y|X=x) = \sum_{y} y Pr(Y=y | X=x) = \sum_{y} y f(y | x) $$
Covariance and correlation:
$$ \sigma_{XY} = Cov(X,Y) = E[(X-\mu_X)(Y-\mu_Y)] \quad \quad  $$ 
$$ \rho_{XY} = corr(X,Y) = \frac{\sigma_{XY}}{\sigma_X \sigma_Y} \quad \text{ where } -1 \leq \rho \leq 1$$

Two random variables are \textit{uncorrelated} if $\rho_{XY}=0$. 

Two random variables are \textit{independent} if $ f(y|x) = f(y)$ for all $x$ and $y$ or equivalently $ E(Y|X) = E(Y)$. \\
\hrule 

\newpage
\underline{Sample Mean Distribution} 
$$E(\bar{X})=\mu \quad \quad \quad Var(\bar{X}) = \sigma^2_{\bar{X}} = \frac{\sigma^2}{n} $$
Sample mean is normally distributed: \vspace{-1em}
\begin{enumerate}
\item When the underlying population is normal, \textbf{or}
\item If the sample size is large, say $n\geq 100$ by the Central Limit Theorem (CLT) 
\end{enumerate}
 
\hrule 
\vspace{0.5em}          
\underline{Confidence Intervals} 

\textit{Known population variance:} \\
$1-\alpha$ confidence interval for the population mean $\mu$:
$$ \bar{x} \pm  \underbrace{z_{\alpha/2}  \frac{\sigma}{\sqrt{n}}}_{\text{Margin of Error}} $$
where $z_{\alpha/2}$ is the $z$-value that leaves area $\alpha/2$ in the upper tail of the standard normal distribution. \\

\textit{Unknown population variance:} \\
$1-\alpha$ confidence interval for the population mean $\mu$:
$$ \bar{x} \pm  t_{n-1,\alpha/2}  \frac{S}{\sqrt{n}} $$
where $t_{n-1,\alpha/2}$ is the $t$-value that leaves area $\alpha/2$ in the upper tail of the t-distribution. $n-1$ is the degrees of freedom. Since $t$ distribution looks just like the standard normal for large $n$, for $n\geq 100$ continue using the standard normal table. \\
\hrule 

\newpage
\underline{Hypothesis Testing} 

Test null hypothesis $ H_0: \mu = \mu_0 $ against alternative hypothesis $H_1: \mu \neq \mu_0$. Construct test statistic $Z$ if true population variance is known, else use $T$-statistic.  
 $$ z_0 = \frac{\bar{x}-\mu_0}{\sigma/\sqrt{n}} \quad \text{and} \quad t_0 = \frac{\bar{x}-\mu_0}{S/\sqrt{n}}  $$
Under the null if $\bar{X} \sim N(\mu_0, \sigma^2/n)$, then $Z \sim N(0,1)$ and $T \sim t_{n-1}$. 
In case of known population variance, reject the null if $|z_0|>z_{\alpha/2}$. In the case of unknown population variance, reject the null if $|t_0|>t_{n-1,\alpha/2}$. When $n \geq 100$ you can reject the null if $|t_0|>z_{\alpha/2}$. \\

\hrule \vspace{0.5em}
\underline{p-value:} 

Known variance: $ p = 2Pr(Z>|z_0|)$ \\
Unknown variance, $n<100$: $p=2Pr(T>|t_0|)$ \\
Unknown variance, $n \geq 100$:  $p=2Pr(Z>|t_0|)$ \\
\hrule \vspace{1em}
Note:\textit{ You will not need to refer to the $t$-table for the exam, only the standard normal table. 
}
\end{document}
