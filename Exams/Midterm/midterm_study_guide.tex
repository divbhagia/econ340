\documentclass{./../../Latex/handout}
\begin{document}
\thispagestyle{plain}
\myheader{Midterm Study Guide}
\rhead{Midterm Study Guide}
\textit{Here is some general information about the midterm exam.}

\begin{itemize}
\item This will be a closed-book exam. You will be given a formula sheet and the normal distribution table to assist you. Both are uploaded on Canvas.
\item You will be allowed to use a calculator. No phones or computers are allowed. 
\item I have also uploaded two practice exams with solutions to give you an idea of what to expect on the exam. 
\item Some of the questions on the exam will be similar to the ones we did in class or on the problem sets. So please review them.
\item The most important thing to prepare for the exam is to review the notes for each topic and the slides and handouts for each lecture. 
\item You do not need to know any proofs for the exam.
\item While writing the exam, make sure to show your work so you can get partial credit in case you make a small error.  
\end{itemize}

\newpage
 \textit{What all do you need to know?}
\begin{enumerate}
\item \textit{Summation Notation} -- It's important to be comfortable with the summation notation as it will enable you to understand and implement formulas for different statistics. 
\item \textit{Describing Data}  
\begin{itemize}
\item How to fill in a frequency distribution table.
\item How to calculate the mean, median, and mode for a variable. (Including calculating mean from the frequency distribution table for grouped data.)
\item What is a percentile?
\item How to calculate a weighted mean. 
\item Means are affected by outliers while medians are not and why that is the case.
\item How to calculate the variance, standard deviation, covariance, and correlation for sample or population data. Understand how and why the formula for these statistics works. 
\item How to calculate the $Z$-score and what it captures.
\item What do different values of correlation imply? 
\item Understand that correlation does not imply causation 
\end{itemize}
\item \textit{Random Variables} 
\begin{itemize}
\item How to calculate the expected value and variance of discrete random variables. What do these capture?
\item How to look up the area under the curve for any normally distributed variable from the standard normal table
\item How to calculate the conditional probability from marginal probabilities and vice-versa
\item Understand and be able to calculate the conditional expectation 
\item What does it mean for random variables to be uncorrelated or independent?
\end{itemize}
\newpage
\item \textit{Sampling and Estimation} 
\begin{itemize}
\item What are the properties of a good estimator?
\item Sample mean is a random variable. 
\item What is the expectation and variance of the sample mean? 
\item When is the sample mean normally distributed? When to invoke the Central Limit Theorem? 
\item Be able to construct and explain the logic behind confidence intervals.
\item Be able to reason what happens to the variance of the sample mean and the margin of error due to changes in population variance or sample size. 
\item Be able to test a hypothesis and explain the logic behind it. (We only covered two-tailed tests i.e. testing null hypothesis of the form $H_0: \mu = \mu_0$ against alternative hypothesis $H_1: \mu \neq \mu_0$.)
\item Be able to calculate and explain what a $p$-value is in simple English. 
\end{itemize}
\end{enumerate}

\end{document}
